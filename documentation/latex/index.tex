\section{Référence}\label{index_intro_0}
{\tt http\-://www710.\-univ-\/lyon1.\-fr/$\sim$bonnev/\-T\-P3.\-html} \section{Potentiel-\/tâche et couplage}\label{index_intro_1}
Pour en terminer avec les graphes, l'idée est de combiner les 2 problèmes. On suppose que pour chaque tâche, il faut une ressource choisie parmi un sous-\/ensemble. Chaque ressource est unique et non partageable (ce peut-\/être un chef-\/d'équipe spécialisé si on fait un chantier, un processeur (ou noeud de supercalculateur) particulier si les tâches sont des processus, une machine particulière si c'est pour une manufacture, ....). Une fois affectée à une tâche, une ressource ne peut pas être préemptée (elle est conservée par la tâche jusqu'à sa terminaison).\section{Résultat attendu}\label{index_intro_2}
À l'issu des calculs, 2 situations possibles \-:

pas d'ordonnancement possible compatible avec les contraintes de couplage \-: un simple message signifiant l'échec est le minimum attendu; la situation de blocage peut aussi être précisée (tâches et ressources impliquées ainsi que les dates utiles).

il existe un ordonnancement compatible avec les contraintes \-: un ordonnacement possible a donc été trouvé, il doit être affiché. Ainsi pour chaque tâche, seront affichées les informations suivantes \-:

date prévue de démarrage, marge restante, ressource affectée, éventuellement la date de fin En l'absence d'autres critères, nous nous contenterons de la première solution trouvée.\section{Variantes}\label{index_intro_3}
Suivant les situations, de nombreuses variantes existes et certaines hypothèses remises en causes. Par exemple, il est possible d'imaginer que les ressources sont préemptibles et les tâches interruptibles (ce peut être le cas par exemple si on considère des affectations de processus sur des ressources particulières), mais souvent il y a des coûts associés (prise en compte du surcoût du à l'interruption, coût et délais si un processus est déplacé d'un processeur à un autre, ...) et les modèles deviennent encore plus complexe.

De même, sur ce problème simple, la dimension économique peut aussi être prise en compte, si on considére des pénalités pour le retard et des coûts associés au ressources \-: faut-\/il augmenter le nombre de ressource, si oui pour quelles tâches, ou payer des pénalités, ou combiner ? 